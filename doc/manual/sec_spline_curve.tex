\section{B-spline Curves}

This section is optional reading for those who want to
become acquainted with some of the mathematics of
B-splines curves. For a description of the data structure for
B-spline curves in SISL, see section \ref{curveobject}.

A B-spline curve is defined by the formula
$$ {\bf c}(t) = \sum_{i=1}^{n} {\bf p}_{i} B_{i,k,{\bf t}}(t). $$
The dimension of the curve ${\bf c}$ is equal to that of its
{\it control points} ${\bf p}_i$. For example, if the dimension of the
control points
is one, the curve is a function, if the dimension is two,
the curve is planar, and if the dimension is three,
the curve is spatial. SISL also allows higher dimensions.

Thus, a B-spline curve is a linear combination of a sequence of B-splines
$B_{i,k,{\bf t}}$ (called a B-basis)
uniquely determined by a knot vector ${\bf t}$ and
the order $k$. Order is equivalent to polynomial degree plus one.
For example, if the order is two, the degree is one and the B-splines
and the curve $c$ they generate are (piecewise) linear.
If the order is three, the degree is two and the B-splines and
the curve are quadratic. Cubic B-splines and
curves have order 4 and degree 3, etc.

The parameter range of a B-spline curve ${\bf c}$ is the interval
$$ [t_k, t_{n+1}], $$
and so mathematically, the curve is a mapping
${\bf c}: [t_k, t_{n+1}] \to \RR^d$, where $d$ is the Euclidean space
dimension of its control points.

The complete representation of a B-spline curve consists of
\begin{description}
\item[$dim$]: The dimension of the underlying Euclidean space,
              $1,2,3,\ldots$.
\item[$n$]: The number of vertices (also the number of B-splines)
\item[$k$]: The order of the B-splines.
\item[${\bf t}$]: The knot vector of the B-splines.
            ${\bf t} = (t_1, t_2, \ldots, t_{n+k})$.
\item[${\bf p}$]: The control points of the B-spline curve.
           $p_{d,i}\;,\; d=1,\ldots,dim\;,\;
                i=1,\ldots,n.\;\;$
                e.g. when $dim = 3$, we have
                ${\bf p} = (x_1,y_1,z_1,x_2,y_2,z_2,\ldots,x_n,y_n,z_n)$.
\end{description}

We note that arrays in $c$ start at index 0 which means,
for example, that if the array $t$ holds the knot vector,
then $t[0] = t_1,\ldots, t[n+k-1] = t_{n+k}$
and the parameter interval goes from
$t[k-1]$ to $t[n]$. Similar considerations apply to the other arrays.

The data in the representation must satisfy certain conditions:
\begin{itemize}
\item The knot vector must be non-decreasing: $t_i \le t_{i+1}$.
      Moreover, two knots $t_i$ and $t_{i+k}$ must be distinct:
      $t_i < t_{i+k}$.
\item The number of vertices should be greater than or equal
      to the order of the curve: $n \ge k$.
\item There should be $k$ equal knots at the beginning and at the end
      of the knot vector; that is the knot vector ${\bf t}$
      must satisfy the conditions $t_1 = t_2 = \ldots = t_k$
      and $t_{n+1} = t_{n+2} = \ldots = t_{n + k}$.
\end{itemize}

To understand the representation better, we will look at
three parts of the
representation: the B-splines (the basis functions),
the knot vector and the control polygon.

\subsection{B-splines}

A set of B-splines is determined by the order $k$
and the knots. For example,
to define a single B-spline of degree one, we need three knots.
In figure~\ref{curve1} the three knots are marked as dots.
Knots can also be equal as shown in figure
\ref{curve2}.
By taking a linear combination of the three types of B-splines shown
in figures~\ref{curve1} and~\ref{curve2}
we can generate a linear spline function
as shown in figure~\ref{curve3}.
\begin{figure}
        \begin{center}
                \begin{picture}(250,80)(0,0)
                \put(80,0){\circle*{2}}
                \put(150,0){\circle*{2}}
                \put(230,0){\circle*{2}}

                \put(30,4){\vector(0,1){76}}
                \put(28,10){\line(1,0){4}}
                \put(28,70){\line(1,0){4}}
                \put(20,10){\makebox(0,0){0.0}}
                \put(20,70){\makebox(0,0){1.0}}

                \thicklines
                \put(80,10){\line(6,5){70}}
                \put(150,70){\line(4,-3){80}}
                \end{picture}\\
        \end{center}
  \caption{\label{curve1}A linear B-spline (order 2) defined by
                        three knots.}
\end{figure}
\begin{figure}
        \begin{center}
                \begin{picture}(300,85)(0,0)
                \put(40,0){\circle*{2}}
                \put(280,0){\circle*{2}}
                \put(120,5){\circle*{2}}
                \put(200,5){\circle*{2}}
                \put(40,5){\circle*{2}}
                \put(280,5){\circle*{2}}

                \put(20,9){\vector(0,1){76}}
                \put(18,15){\line(1,0){4}}
                \put(18,75){\line(1,0){4}}
                \put(10,15){\makebox(0,0){0.0}}
                \put(10,75){\makebox(0,0){1.0}}

                \put(170,9){\vector(0,1){76}}
                \put(168,15){\line(1,0){4}}
                \put(168,75){\line(1,0){4}}
                \put(160,15){\makebox(0,0){0.0}}
                \put(160,75){\makebox(0,0){1.0}}

                \thicklines
                \put(120,15){\line(-4,3){80}}
                \put(200,15){\line(4,3){80}}
                \end{picture}\\
        \end{center}
  \caption{\label{curve2}Linear B-splines of with multiple knots at one end.}
\end{figure}
\begin{figure}
        \begin{center}
                \begin{picture}(300,90)(0,0)
                \put(0,0){\circle*{2}}
                \put(0,5){\circle*{2}}
                \put(60,5){\circle*{2}}
                \put(120,5){\circle*{2}}
                \put(160,5){\circle*{2}}
                \put(240,5){\circle*{2}}
                \put(300,5){\circle*{2}}
                \put(300,0){\circle*{2}}

                \multiput(0,10)(12,10){5}{\line(6,5){10}}
                \multiput(60,10)(-12,8){5}{\line(-3,2){10}}
                \multiput(60,10)(12,2){5}{\line(6,1){10}}
                \multiput(120,10)(-12,10){5}{\line(-6,5){10}}
                \multiput(120,10)(12,6){3}{\line(2,1){10}}
                \multiput(160,10)(-12,3){3}{\line(-4,1){10}}
                \multiput(160,10)(12,12){6}{\line(1,1){10}}
                \multiput(240,10)(-12,3){6}{\line(-4,1){10}}
                \multiput(240,10)(12,12){5}{\line(1,1){10}}
                \multiput(300,10)(-12,16){5}{\line(-3,4){10}}

                \thicklines
                \put(0,50){\line(6,1){60}}
                \put(60,60){\line(3,-2){60}}
                \put(120,20){\line(4,1){40}}
                \put(160,30){\line(4,3){80}}
                \put(240,90){\line(3,-1){60}}
                \end{picture}\\
        \end{center}
  \caption{\label{curve3}A B-spline curve of dimension 1 as a linear
            combination of a sequence of B-splines.
            Each B-spline (dashed) is scaled by a coefficient.}
\end{figure}

A quadratic B-spline is a linear combination of two linear
B-splines. Shown in figure~\ref{curve4}
is a quadratic B-spline defined by four knots.
A quadratic B-spline is
the sum of two products, the first product between the linear B-spline
on the left and a corresponding line from 0 to 1,
the second product between the linear B-spline
on the right and a corresponding line from 1 to 0;
see figure~\ref{curve4}.
For higher degree B-splines there is a similar definition.
A B-spline of order $k$ is the sum of two B-splines of
order $k-1$, each weighted with weights in the interval [0,1].
In fact we define B-splines of order 1 explicitly as box functions,
$$  B_{i,1}(t) =  \cases{1 & if $t_i \le t < t_{i+1}$; \cr
                              0 &  otherwise, \cr } $$
and then the complete definition of a $k$-th order B-spline is
$$ B_{i,k}(t) = {t - t_i \over t_{i+k-1} - t_i} B_{i,k-1}(t)
             +
             {t_{i+k} - t \over t_{i+k} - t_{i+1}} B_{i-1,k-1}(t). $$

\begin{figure}
        \begin{center}
                \begin{picture}(250,70)(0,0)

                \put(10,4){\vector(0,1){76}}
                \put(8,10){\line(1,0){4}}
                \put(8,70){\line(1,0){4}}
                \put(0,10){\makebox(0,0){0.0}}
                \put(0,70){\makebox(0,0){1.0}}

                \put(35,0){\circle*{2}}
                \put(95,0){\circle*{2}}
                \put(155,0){\circle*{2}}
                \put(215,0){\circle*{2}}

                \multiput(35,10)(18,9){7}{\line(2,1){10}}
                \multiput(215,10)(-18,9){7}{\line(-2,1){10}}

                \thicklines
                \put(35,10){\line(1,1){60}}
                \put(95,10){\line(1,1){60}}
                \put(155,10){\line(-1,1){60}}
                \put(215,10){\line(-1,1){60}}

                \bezier{70}(35,10)(65,10)(95,40)
                \bezier{70}(95,40)(125,70)(155,40)
                \bezier{70}(215,10)(185,10)(155,40)
                \end{picture}\\
        \end{center}
  \caption{\label{curve4}A quadratic B-spline, the two linear
           B-splines and the corresponding lines (dashed)
           in the quadratic B-spline definition.}
\end{figure}

B-splines satisfy some important properties for curve and surface design.
Each B-spline is non-negative and it can be shown that they sum to
one,
$$ \sum_{i=1}^{n}B_{i,k,{\bf t}}(t) = 1. $$
These properties combined mean that B-spline curves
satisfy the {\it convex hull property}: the curve lies in the convex
hull of its control points.
Furthermore, the support of the B-spline $B_{i,k,{\bf t}}$ is
the interval $[t_i,t_{i+k}]$ which means that B-spline
curves has {\it local control}: moving one control point only
alters the curve locally.

Due to the demand of $k$ multiple knots at the ends of the knot
vector, B-spline curves in SISL also have the {\it endpoint property}:
the start point of the B-spline curve
equals the first control point and the end point equals the
last control point, in other words
$$ {\bf c}(t_k) = {\bf p}_1
     \qquad \hbox{and} \qquad
   {\bf c}(t_{n+1}) = {\bf p}_n. $$

\subsection{\label{contrlpoly}The Control Polygon}

The control points ${\bf p}_i$ define the vertices
The {\it control polygon} of a B-spline curve is the polygonal
arc formed by its control points,
${\bf p}_0, {\bf p}_1, \ldots,{\bf p}_n$.
This means that
the control polygon, regarded as a parametric curve,
is itself piecewise linear B-spline curve (order two).
If we increase the order, the distance between the control polygon
and the curve increases (see figure~\ref{curve5}).
A higher order B-spline curve tends to smooth
the control polygon and at the same time mimic its shape.
For example, if the control polygon is convex, so is the B-spline curve.

Another property of the control polygon is that it will get closer
to the curve if it is redefined by inserting knots into the curve
and thereby increasing the number of vertices;
see figure~\ref{curve6}.
If the refinement is infinite then the control polygon converges to the curve.
\begin{figure}
        \begin{center}
                \begin{picture}(240,80)(0,0)
                \thicklines
                \put(0,40){\line(2,1){80}}
                \put(80,80){\line(1,-1){80}}
                \put(160,0){\line(1,1){80}}

                \bezier{120}(0,40)(80,80)(120,40)
                \bezier{120}(120,40)(160,0)(240,80)

                \bezier{120}(0,40)(60,70)(120,40)
                \bezier{120}(120,40)(180,20)(240,80)

                \end{picture}\\
        \end{center}
  \caption{\label{curve5}Linear, quadratic, and cubic B-spline
           curves sharing the same control polygon. The control polygon is
                equal to the linear B-spline curve. The curves
                are planar, i.e. the space dimension is two.}
\end{figure}

\begin{figure}
        \begin{center}
                \begin{picture}(240,80)(0,0)
                \thicklines
                \put(0,40){\line(2,1){40}}
                \put(40,60){\line(5,0){40}}
                \put(80,60){\line(2,-1){60}}
                \put(140,30){\line(6,1){60}}
                \put(200,40){\line(1,1){40}}

                \bezier{120}(0,40)(60,70)(120,40)
                \bezier{120}(120,40)(180,20)(240,80)

                \end{picture}\\
        \end{center}
  \caption{\label{curve6}The cubic B-spline curve with a
                redefined knot vector.}
\end{figure}

\subsection{The Knot Vector}

The knots of a B-spline curve describe the following properties of the curve:
\begin{itemize}
\item The parameterization of the B-spline curve
\item The continuity at the joins between the adjacent polynomial
   segments of the B-spline curve.
\end{itemize}
In figure~\ref{curve7} we have two curves
with the same control polygon and order but
with different parameterization.

This example is not meant as an encouragement to use
parameterization for modelling, rather to make users
aware of the effect of parameterization. Something close to
curve length parameterization is in most cases
preferable. For interpolation, chord-length parameterization
is used in most cases.

The number of equal knots determines the degree
of continuity. If $k$ consecutive internal knots are equal,
the curve is discontinuous.
Similarly if $k-1$ consecutive internal knots are equal,
the curve is continuous but not in general differentiable.
A continuously differentiable curve
with a discontinuity in the second derivative
can be modelled using $k-2$ equal knots etc. (see figure~\ref{curve8}).
Normally, B-spline curves in SISL are expected to be continuous.
For intersection algorithms, curves are usually expected to
be continuously differentiable ($C^1$).

\begin{figure}
        \begin{center}
                \begin{picture}(240,240)(0,0)
                \thicklines
                \put(0,130){\circle*{2}}
                \put(0,135){\circle*{2}}
                \put(0,140){\circle*{2}}
                \put(240,130){\circle*{2}}
                \put(240,135){\circle*{2}}
                \put(240,140){\circle*{2}}

                \put(60,140){\circle*{2}}
                \put(180,10){\circle*{2}}


                \put(0,0){\circle*{2}}
                \put(0,5){\circle*{2}}
                \put(0,10){\circle*{2}}
                \put(240,0){\circle*{2}}
                \put(240,5){\circle*{2}}
                \put(240,10){\circle*{2}}

                \put(0,185){\line(2,1){80}}
                \put(80,225){\line(1,-1){80}}
                \put(160,145){\line(1,1){80}}

                \put(0,55){\line(2,1){80}}
                \put(80,95){\line(1,-1){80}}
                \put(160,15){\line(1,1){80}}

                \bezier{130}(0,185)(80,225)(100,205)
                \bezier{160}(100,205)(160,145)(240,225)

                \bezier{160}(0,55)(80,95)(140,35)
                \bezier{130}(140,35)(160,15)(240,95)
                \end{picture}\\
        \end{center}
  \caption{\label{curve7}Two quadratic B-spline curves with the same
control polygon but different knot vectors. The curves and the control
polygons are two-dimensional.}
\end{figure}

\begin{figure}
        \begin{center}
                \begin{picture}(300,90)(0,0)
                \thicklines
                \put(0,0){\circle*{2}}
                \put(0,5){\circle*{2}}
                \put(0,10){\circle*{2}}

                \put(150,5){\circle*{2}}
                \put(150,10){\circle*{2}}

                \put(300,0){\circle*{2}}
                \put(300,5){\circle*{2}}
                \put(300,10){\circle*{2}}

                \bezier{200}(0,50)(80,90)(150,20)
                \bezier{200}(150,20)(220,80)(300,90)
                \end{picture}\\
        \end{center}
  \caption{\label{curve8}A quadratic B-spline curve with two
equal internal knots.}
\end{figure}

\subsection{NURBS Curves}

A NURBS (Non-Uniform Rational B-Spline) curve is a generalization
of a B-spline curve,
$$ {\bf c}(t) = {\sum_{i=1}^{n} w_i {\bf p}_{i} B_{i,k,{\bf t}}(t)
                 \over
                 \sum_{i=1}^{n} w_i B_{i,k,{\bf t}}(t)} . $$
In addition to the data of a B-spline curve, the NURBS curve
${\bf c}$ has a sequence of weights $w_1,\ldots,w_n$.
One of the advantages of NURBS curves over B-spline curves is that
they can be used to represent conic sections exactly (taking the
order $k$ to be three).
A disadvantage is that NURBS curves depend nonlinearly on their weights,
making some calculations, like the evaluation of derivatives,
more complicated and less efficient than with B-spline curves.

The representation of a NURBS curve is the same as for a B-spline
except that it also includes
\begin{description}
\item[${\bf w}$]: A sequence of weights
            ${\bf w} = (w_1, w_2, \ldots, w_n)$.
\end{description}

In SISL we make the assumption that
\begin{itemize}
\item The weights are (strictly) positive: $w_i > 0$.
\end{itemize}

Under this condition, a NURBS curve, like its B-spline cousin,
enjoys the convex hull property. 
Due to $k$-fold knots at the ends of the knot vector,
NURBS curves in SISL alos have the endpoint 

