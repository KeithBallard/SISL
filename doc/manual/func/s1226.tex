\subsection{Evaluate position, first derivative, curvature and radius of
               curvature of a curve at a given parameter value, from the
               right hand side.}
\funclabel{s1226}
\begin{minipg1}
Evaluate position, derivatives, curvature and radius of
               curvature of a curve at a given parameter value, from the
               right hand side.
\end{minipg1} \\ \\
SYNOPSIS\\
        \> void s1226(\begin{minipg3}
            {\fov curve}, {\fov der}, {\fov parvalue}, {\fov leftknot}, {\fov derive}, {\fov curvature},
	   radius\_of\_{\fov curvature}, {\fov jstat})
                \end{minipg3}\\
                \>\>    SISLCurve    \>  *{\fov curve};\\
                \>\>    int    \>  {\fov der};\\
                \>\>    double \> parvalue;\\
                \>\>    int    \>  *{\fov leftknot};\\
                \>\>    double \> derive[\,];\\
                \>\>    double \> curvature[\,];\\
                \>\>    double \>*{\fov radius}\_of\_curvature;\\
                \>\>    int    \>  *{\fov jstat};\\
\\
ARGUMENTS\\
	\>Input Arguments:\\
        \>\>    {\fov curve}\> - \>  \begin{minipg2}
                     Pointer to the curve for which position
                       and derivatives are to be computed.
                               \end{minipg2}\\
        \>\>    {\fov der}\> - \> The number of derivatives to compute.\\
	            \>\>\>\>\>       $< 0$ : Error.\\
		    \>\>\>\>\>       $= 0$ : Compute position.\\
		    \>\>\>\>\>       $= 1$ : Compute position and first derivative.\\
		    \>\>\>\>\>       etc.\\
        \>\>    {\fov parvalue}\> - \>  \begin{minipg2}
                     The parameter value at which to compute
                       position and derivatives.
                               \end{minipg2}\\
\\
	\>Input/Output Arguments:\\
        \>\>    {\fov leftknot}\> - \>  \begin{minipg2}
                     Pointer to the interval in the knot vector
                        where ax is located. If et is the knot vector,
                        the relation
                          $$et[ileft] < parvalue <= et[ileft+1]$$
                        should hold. (If parvalue = et[ik-1] then ileft
                        should be ik-1. Here in is the number of B-spline
                        coefficients.)
                        If ileft does not have the right value upon
                        entry to the routine, its value will be changed
                        to the value satisfying the above condition.
                               \end{minipg2}\\
\\
	\>Output Arguments:\\
        \>\>    {\fov derive}\> - \>  \begin{minipg2}
                     Double array of dimension [(ider+1)*idim]
                       containing the position and derivative vectors.
                       (idim is the number of components of each B-spline
                       coefficient, i.e. the dimension of the Euclidean
                       space in which the curve lies.)
                       These vectors are stored in the following order:
                       First the idim components of the position vector,
                       then the idim components of the tangent vector,
                       then the idim components of the second derivative
                       vector, and so on.
                       (The C declaration of eder as a two dimensional array
                       would therefore be eder[ider+1,idim].)
                               \end{minipg2}\\
        \>\>    {\fov curvature}\> - \>  \begin{minipg2}
                     Array of dimension idim
                               \end{minipg2}\\
        \>\>    {\fov radius}\> - \>  \begin{minipg2}
                    1, indicates
                       that the radius of curvature is infinit.
                               \end{minipg2}\\
        \>\>    {\fov jstat}\> - \> Status messages\\ 
                         \>\>\>\>\>      $> 0$      : Warning.\\
                         \>\>\>\>\>      $= 0$      : Ok.\\
                         \>\>\>\>\>      $< 0$      : Error.\\
\\
EXAMPLE OF USE\\
		\>      \{ \\

                \>\>    SISLCurve    \>  *{\fov curve}; \, /* Must be defined */\\
                \>\>    int    \>  {\fov der} = 1;\\
                \>\>    double \> parvalue; \, /* Must be defined */\\
                \>\>    int    \>  {\fov leftknot} = 0; /* Define initially as zero. For consequtive evaluations \\
                \>\>\>\>\>\> leave leftknot as returned from s1226 */\\
                \>\>    double \>derive[6]; /* Geometry space dimension times (der + 1) */\\
                \>\>    double \>curvature[3]; /* Geometry space dimension */\\
                \>\>    double \> {\fov radius}\_of\_curvature = 0;\\
                \>\>    int    \>  {\fov jstat} = 0;\\                \>\>    \ldots \\
        \>\>s1226(\begin{minipg4}
            {\fov curve}, {\fov der}, {\fov parvalue}, {\fov leftknot}, {\fov derive}, {\fov curvature},
	   \&radius\_of\_{\fov curvature}, \&{\fov jstat});
                \end{minipg4}\\
                \>\>    \ldots \\
		\>      \}
\end{tabbing}
