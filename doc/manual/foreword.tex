\chapter{Preface}
Welcome to the SISL 4.5 user's manual.  SISL stands for
\emph{\textbf{Si}ntef \textbf{S}pline \textbf{L}ibrary}, and has been gradually 
developed and enhanced for more than three decades by the geometry group at SINTEF in Oslo.
Although it is very comprehensive, its organisation is simple.  There are but a 
few structures, and its nearly four hundred main functions can usually be employed
directly and individually. This manual organises and explains the main routines.  However, much of this
information can also be found directly in the code in the form of commentaries.

The complete software package you have in your hands should contain the following:
\begin{itemize}
\item The SISL 4.5 distribution and reference guide (the document you are reading now)
\item Supplementary routines for writing SISL objects to streams (including file 
streams) in a simple ASCII format called \textbf{\Verb/Go/}.
\item A selection of \emph{sample programs}, designed to demonstrate functionalities
and use of SISL.
\item Source code for a simple \emph{viewer} that can be used to view geometric objects stored
in the \Verb/Go/-format.  This allows visual inspection of SISL-generated curves
and surfaces, as well as points.
\end{itemize}

\section{The structure of this document}
\textbf{Chapter 2} is a general introduction to SISL and its programming style. A simple example program
including instructions in how to compile and link the program and the expected output is provided.
Since it is strongly recommended that the user has some general knowledge of splines, this 
chapter also contains a couple of sections introducing the subject of spline curves and
surfaces. \\
\\
\textbf{Chapter 3 to 11} presents the main SISL routines.  \\
\\
\textbf{Chapter 12} goes through the provided sample programs and explain what these do,
and what the user can expect to learn from them.  There are a total of 15 sample 
programs, ranging from very basic to intermediate complexity.\\
\\
The goal of \textbf{Chapter 13} is to explain the use of the \emph{viewer program}, 
which is a small but handy tool for visually inspecting results from SISL routines.\\
\\
\textbf{Chapter 14} is an appendix presenting an explanation of the error codes used in SISL.\\
\\
Finally there is an \textbf{annex}, citing the text of the General Public License.

\section{The structure of the software package}\label{compile}
There are seven directories:
\begin{itemize}
\item \textbf{\Verb-include/-} - the inlude files related to the 4.5 release of SISL.
\item \textbf{\Verb-src/-} - the source code of the 4.5 release of SISL.
\item \textbf{\Verb-doc/-} - the basis for this document. 
\item \textbf{\Verb-streaming/-} - source code for the routines that can read and write
SISL objects to a stream.
\item \textbf{\Verb-examples/-} - sample programs making use of the SISL 4.5 source code.
\item \textbf{\Verb-viewer/-} - source code for a viewer that can be used to view SISL 
  objects saved in the \verb/Go/-format.
\item \textbf{\Verb-app/-} - the expected directory for test programs and applications. A
  couple of applications are provided including the example program described in Chapter 2.
\end{itemize}
Furthermore is the file CMakeLists.txt provided to facilitate building the library.

\section{Licensing information}
SISL is distributed under the \emph{GNU Affero General Public License} (aGPL).  The license text 
is given in its entirety as an annex to this document.  Commercial licenses are also
available from SINTEF.  You can contact Tor Dokken (tor.dokken@sintef.no) for more
information.

%% *what is included in this distibution
%% *explain chapters
%% *the manual
%% *licensing information
%% *compilation
%% *the object viewer
