\section{\label{curveobject}Curve Object}

In the library both B-spline and NURBS curves are stored in a struct SISLCurve containing
the following:
\input{type/SISLCurve}
Note that in the rational case are the curve coefficients stored as
\newline
${w_1  {\bf p}_1, w_1, w_2 {\bf p}_2, w_2, \ldots,
  w_n  {\bf p}_n, w_n}$ where $w_i$ are the weights and ${\bf p}_i$, $i=1, \ldots, n$ are the
curve coefficients.

When using a curve, do not declare a SISLCurve but a pointer to a SISLCurve,
and initialize it to point on NULL. Then you may use the
dynamic allocation functions newCurve and freeCurve described below,
to create and delete curves.

There are two ways to pass coefficient and knot arrays
to newCurve.
By setting $icopy=1$, newCurve
allocates new arrays and copies the given ones.
But by setting $icopy=0$ or 2, newCurve simply points
to the given arrays. Therefore it is IMPORTANT that the
given arrays have been allocated in free memory beforehand.

\pgsbreak
\subsection{Create new curve object.} \label{sec:newCurve}
\funclabel{newCurve}
\begin{minipg1}
  Create and initialize a SISLCurve-instance. Note that the vertex input to a
  rational curve is unstandard. Given the curve
  $$
{\bf c}(t) = {\sum_{i=1}^{n} w_i {\bf p}_{i} B_{i,k,{\bf t}}(t)
                 \over
                 \sum_{i=1}^{n} w_i B_{i,k,{\bf t}}(t)},
$$
must the vertices be given as
${w_1  {\bf p}_1, w_1, w_1  {\bf p}_2, w_2, \ldots,
  w_n  {\bf p}_n, w_n}$ when invoking this function. Thus the vertices are multiplied with the
associated weight.
\end{minipg1} \\ \\
SYNOPSIS\\
        \>SISLCurve *newCurve(\begin{minipg3}
        {\fov number}, {\fov order}, {\fov knots}, {\fov coef}, {\fov kind}, {\fov dim}, {\fov copy})
                \end{minipg3}\\[0.3ex]
                \>\>    int    \>       {\fov number};\\
                \>\>    int    \>       {\fov order};\\
                \>\>    double \>       {\fov knots}[\,];\\
                \>\>    double \>       {\fov coef}[\,];\\
                \>\>    int    \>       {\fov kind};\\
                \>\>    int    \>       {\fov dim};\\
                \>\>    int    \>       {\fov copy};\\
\\
ARGUMENTS\\
        \>Input Arguments:\\
        \>\>    {\fov number}   \> - \> Number of vertices in the new curve.\\
        \>\>    {\fov order} \> - \> Order of curve.\\
        \>\>    {\fov knots} \> - \> Knot vector of curve.\\
        \>\>    {\fov coef}  \> - \> \begin{minipg2}
                      Vertices of curve. These can either be the $dim$
                      \mbox{dimensional}
                      non-rational vertices, or the $(dim+1)$ dimensional rational
                      vertices.
                                     \end{minipg2}\\[0.8ex]
        \>\>    {\fov kind} \> - \> Type of curve.\\
        \>\>\>\>\>       $= 1$ :\> Polynomial B-spline curve.\\
        \>\>\>\>\>       $= 2$ :\> Rational B-spline (nurbs) curve.\\
        \>\>\>\>\>       $= 3$ :\> Polynomial Bezier curve.\\
        \>\>\>\>\>       $= 4$ :\> Rational Bezier curve.\\
        \>\>    {\fov dim} \> - \> Dimension of the space in which the
                                   curve lies.\\
        \>\>    {\fov copy} \> - \> Flag \\
        \>\>\>\>\>       $= 0$ :\> Set pointer to input arrays.\\
        \>\>\>\>\>       $= 1$ :\> Copy input arrays.\\
        \>\>\>\>\>       $= 2$ :\> Set pointer and remember to free arrays.\\
\\
        \>Output Arguments:\\
        \>\>    {\fov newCurve} \> - \> \begin{minipg2}
                                 Pointer to the new curve. If it is impossible
                                 to allocate space for the curve, newCurve
                                 returns NULL.
                                \end{minipg2}\\
\newpagetabs
EXAMPLE OF USE\\
        \>      \{ \\
        \>\>    SISLCurve    \> *{\fov curve} = NULL;\\
        \>\>    int    \>       {\fov number} = 10;\\
        \>\>    int    \>       {\fov order} = 4;\\
        \>\>    double \>       {\fov knots}[14];\\
        \>\>    double \>       {\fov coef}[30];\\
        \>\>    int    \>       {\fov kind} = 1;\\
        \>\>    int    \>       {\fov dim} = 3;\\
        \>\>    int    \>       {\fov copy} = 1;\\
        \>\>    \ldots \\
        \>\>{\fov curve} = newCurve(\begin{minipg4}
          {\fov number}, {\fov order}, {\fov knots}, {\fov coef}, {\fov kind},
          {\fov dim}, {\fov copy});
        \end{minipg4}\\
        \>\>    \ldots \\
        \>      \} \\
\end{tabbing}

\pgsbreak
\subsection{Make a copy of a curve.}
\funclabel{copyCurve}
\begin{minipg1}
Make a copy of a curve.
\end{minipg1}\\ \\
SYNOPSIS\\
        \>SISLCurve *copyCurve(\begin{minipg3}
          {\fov pcurve})
        \end{minipg3}\\[0.3ex]
        \>\>    SISLCurve \> *{\fov pcurve};\\
\\
ARGUMENTS\\
        \>Input Arguments:\\
        \>\>    {\fov pcurve}    \> - \> Curve to be copied.\\
\\
        \>Output Arguments:\\
        \>\>    {\fov copyCurve} \> - \> The new curve.\\
\\
EXAMPLE OF USE\\
        \>      \{ \\
        \>\>    SISLCurve \> *{\fov curvecopy} = NULL;\\
        \>\>    SISLCurve \> *{\fov curve} = NULL;\\
        \>\>    int       \> {\fov number} = 10;\\
        \>\>    int       \> {\fov order} = 4;\\
        \>\>    double    \> {\fov knots}[14]; \,/* Must be defined */\\
        \>\>    double    \> {\fov coef}[30]; \, /* Must be defined */\\
        \>\>    int       \> {\fov kind} = 1; /* Non-rational */ \\
        \>\>    int       \> {\fov dim} = 3;\\
        \>\>    int       \> {\fov copy} = 1;\\
        \>\>    \ldots \\
        \>\>curve = newCurve(\begin{minipg4}
          {\fov number}, {\fov order}, {\fov knots}, {\fov coef}, {\fov kind}, {\fov dim}, {\fov copy});
        \end{minipg4}\\
        \>\>    \ldots \\
        \>\>curvecopy = copyCurve(\begin{minipg4}
          {\fov curve});
        \end{minipg4}\\
        \>\>    \ldots \\
        \>      \}
\end{tabbing}

\pgsbreak
\subsection{Delete a curve object.} \label{sec:freeCurve}
\funclabel{freeCurve}
\begin{minipg1}
Free the space occupied by the curve. Before using freeCurve, make sure the curve object exists.
\end{minipg1} \\ \\
SYNOPSIS\\
        \>void freeCurve(\begin{minipg3}
          {\fov curve})
        \end{minipg3}\\[0.3ex]
        \>\>    SISLCurve    \> *{\fov curve};\\
\\
ARGUMENTS\\
        \>Input Arguments:\\
        \>\>    {\fov curve}    \> - \> Pointer to the curve to delete.\\
EXAMPLE OF USE\\
        \>      \{ \\
        \>\>    SISLCurve    \> *{\fov curve} = NULL;\\
        \>\>    int    \>       {\fov number} = 10;\\
        \>\>    int    \>       {\fov order} = 4;\\
        \>\>    double \>       {\fov knots}[14];\\
        \>\>    double \>       {\fov coef}[30];\\
        \>\>    int    \>       {\fov kind} = 1;\\
        \>\>    int    \>       {\fov dim} = 3;\\
        \>\>    int    \>       {\fov copy} = 1;\\
        \>\>    \ldots \\
        \>\>curve = newCurve(\begin{minipg4}
          {\fov number}, {\fov order}, {\fov knots}, {\fov coef}, {\fov kind}, {\fov dim}, {\fov copy});
        \end{minipg4}\\
        \>\>    \ldots \\
        \>\>freeCurve(\begin{minipg4}
          {\fov curve});
        \end{minipg4}\\
        \>\>    \ldots \\
        \>      \}
\end{tabbing}

